\documentclass[letterpaper,12pt]{article}
% packages used
    \usepackage{natbib}
	\usepackage{threeparttable}
	\usepackage[format=hang,font=normalsize,labelfont=bf]{caption}
	\usepackage{amsmath}
	\usepackage{amssymb}
	\usepackage{amsthm}
	\usepackage{caption}
	\usepackage{subcaption}
	\usepackage{setspace}
	\usepackage{float,color}
	\usepackage[pdftex]{graphicx}
	\usepackage{hyperref}
	\usepackage{multirow}
	\usepackage{float,graphicx,color}
	\usepackage{graphics}
    \usepackage{placeins}
    \usepackage{authblk}
    \usepackage{tikz}

% other setup
	\hypersetup{colorlinks, linkcolor=red, urlcolor=blue, citecolor=red, hypertexnames=false}
	\graphicspath{{./figures/}}
	\DeclareMathOperator*{\Max}{Max}
	\bibliographystyle{aer}
	\numberwithin{equation}{section}
	\numberwithin{figure}{section}
	\numberwithin{table}{section}
	\newcommand\ve{\varepsilon}
	\def\chk{\tikz\fill[scale=0.4](0,.35) -- (.25,0) -- (1,.7) -- (.25,.15) -- cycle;} 
	\newcommand*{\addheight}[2][.5ex]{%
        \raisebox{0pt}[\dimexpr\height+(#1)\relax]{#2}%
        }


\begin{document}


\begin{titlepage}
	\title{Using an Overlapping Generations Model to Assess the Effects of Financing a Large Government Expenditure
	\thanks{The views expressed in this paper are the authors' and should not be interpreted as those of the Congressional Budget Office.}}

    \author[1]{Jaeger Nelson}
	\author[1]{Kerk L. Phillips}

	\affil[1]{\footnotesize US Congressional Budget Office, Washington, DC, USA}

	\date{October 16, 2020\\
	\small{version 2020.10.a}}
	
	\maketitle

	\vspace{-0.3 in}
	\begin{abstract}
	\small{
	This paper ... 

	\vspace{0.1 in}

	\textit{keywords:} economic policy, taxes, overlapping generations, computational economics

	\vspace{0.1 in}

	\textit{JEL classifications: ??} }
	\end{abstract}

	\centering
	IN PROGRESS
	\thispagestyle{empty}
\end{titlepage}


\begin{spacing}{1.5}

\section{Introduction} \label{sec_intro}

	This paper serves two major purposes.  First, it documents the effects of various financing methods for funding a major increase in government expenditures.  Second, it details and documents a substantive revision to CBO's overlapping generations model.

	\subsection{Literature Review} \label{sec_intro_lit}

		Cite the following: \citet{AuerbachKotlikoff1987}, \citet{NishiyamaSmetters:2007}, \citet{NishiyamaReichling:2015}, \citet{NelsonEtal:2019}.


\section{The Overlapping Generations Model} \label{sec_OLGModel}

	This section details CBO's revised overlapping generations model.

	\subsection{Firms' Problem} \label{sec_OLGModel_firms}

	\subsection{Households' Problem} \label{sec_OLGModel_HH}

	\subsection{Government} \label{sec_OLGModel_govt}

	\subsection{Calibration} \label{sec_OLGModel_calib}

	\subsection{Solution Method} \label{sec_OLGModel_soln}


\section{Results from Financing Options} \label{sec_results}

	CBO paper examines three financing mechanisms:

	\begin{itemize}
		\item Tax on labor income only
		\item Tax on combined labor and capital income
		\item Consumption tax
	\end{itemize}

	For this paper do we want to consider other mechanisms as well?
	\begin{itemize}
		\item Lump-sum tax
		\item Other?
	\end{itemize}

\section{Conclusion} \label{sec_concl}

	

\clearpage

% \newpage
% \renewcommand{\theequation}{A.\arabic{equation}}
%                                                  % redefine the command that creates the section number
% \renewcommand{\thesection}{A\arabic{section}}   % redefine the command that creates the equation number
% \setcounter{equation}{0}                         % reset counter
% \setcounter{section}{0}                          % reset section number
% \section*{APPENDICES}                              % use *-form to suppress numbering
% \numberwithin{figure}{section}
% \numberwithin{equation}{section}


\end{spacing}
\newpage
\nocite{*}
\bibliography{EffectsFinancing}

\end{document}